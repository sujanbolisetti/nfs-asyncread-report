\section{Conclusions}
\label{conc}
We proposed a novel approach for asynchronous read operation in NFSv4.1 using callbacks. We observed that current asynchronous I/O operations on an NFS mounted device are triggered sequentially. This blocks the NFS client until the actual data is transferred back from the server. We thus implemented Asynchronous READ operation to improve the throughput of the client by freeing it immediately after recieving a \textsc{nfs4\_ok} from the server. We found out that the overhead caused by the additional callback to the client is minimal and thus performance degradation is minor even in the case of single read. We identified that there is a performance gain on performing asynchronous reads on same file mainly at consecutive offsets. This is because of the decreased seek time on the server as mentioned in section ~\ref{sec:Evaluation}. We also noted that there is a significant throughput gain depending on the size of the data.  

These findings are not specific to our hardware setup. All the observations presented above are agonastic to the network speeds.  

\paragraph{Future Work}

In this paper we have designed and implemented asynchronous read operation. Similarly we can implement asynchronous write operation in NFS. As write operation is not completely synchronous as read operation, hence we feel we may not observe similar benefits to asynchronous read. But nevertheless we may observe some interesting results in case of simultaneous writes.   

%%%%%%%%%%%%%%%%%%%%%%%%%%%%%%%%%%%%%%%%%%%%%%%%%%%%%%%%%%%%%%%%%%%%%%%%%%%%%%
%% For Emacs:
% Local variables:
% fill-column: 70
% End:
%%%%%%%%%%%%%%%%%%%%%%%%%%%%%%%%%%%%%%%%%%%%%%%%%%%%%%%%%%%%%%%%%%%%%%%%%%%%%%
%% For Vim:
% vim:textwidth=70
%%%%%%%%%%%%%%%%%%%%%%%%%%%%%%%%%%%%%%%%%%%%%%%%%%%%%%%%%%%%%%%%%%%%%%%%%%%%%%
% LocalWords:
