\begin{abstract}

NFS is a highly popular protocol that allows fast and seamless sharing of files across the network. Using NFS, remote file systems can be mounted very much the way local file systems are mounted. Most of the I/O operations on a NFS mounted device employ similar mechanisms to that of a local device. But the behaviour of an asynchronous read differ significantly. Asynchronous I/O requests are triggered sequentially on a NFS mounted device, waiting for the reply to each of the requests before triggering the next request. This unnecessarily  blocks the NFS client until the actual data is received.
\newline
\\ We propose a novel asynchronous read operation, where the  server simply acknowledge the receipt of the request. This way clients are not blocked until the actual data is received. Therefore the client can continue processing, while several I/O operations are performed in the background without having to block for the completion of I/O. This enables applications to overlap their compute and I/O processing to improve throughput on a per process basis. Also the client can trigger multiple I/O requests concurrently. The server then sends the actual data in a separate callback channel. Callbacks is one of the several new features added as part of NFSv4.0. Callbacks allow servers to make an RPC directed at the client. Right now callbacks provide an efficient way for delegating a file to the client by avoiding repeated requests to the server in the absence of inter-client conflicts. Here we use callbacks to support true asynchronous I/O, where the server can respond to the client requests after fetching the data from the disk. We have implemented asynchronous read operation using callbacks on NFS-Ganesha and have tested it using Pynfs as the client.
\hfill \break \newline
\noindent\textbf{Acronyms}\hspace{2mm}  NFS (Network File System), RPC (Remote Procedure Call)

\end{abstract}

%%%%%%%%%%%%%%%%%%%%%%%%%%%%%%%%%%%%%%%%%%%%%%%%%%%%%%%%%%%%%%%%%%%%%%%%%%%%%%
%% For Emacs:
% Local variables:
% fill-column: 70
% End:
%%%%%%%%%%%%%%%%%%%%%%%%%%%%%%%%%%%%%%%%%%%%%%%%%%%%%%%%%%%%%%%%%%%%%%%%%%%%%%
%% For Vim:
% vim:textwidth=70
%%%%%%%%%%%%%%%%%%%%%%%%%%%%%%%%%%%%%%%%%%%%%%%%%%%%%%%%%%%%%%%%%%%%%%%%%%%%%%
% LocalWords:
