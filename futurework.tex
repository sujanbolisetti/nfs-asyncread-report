\section{Future Work}
\subsection{Slotid reuse in Asynchronous Read}

NFSv4.1 supports exactly once semantics model. This means that the request is processed by the NFS server once and only once. This also helps the server in identifying the duplicate requests. NFS server does this identification with the help of \textsc{slotid} and \textsc{seqid} that will be present for each request. If another request uses the same \textsc{slotid} and \textsc{seqid} which the server has already seen in the session then that is a duplicate request and then searches in the reply cache to check if it has already processed. If so, it replies the same reply to the client else it will reply current status of the previous request. In normal \textsc{nfs4\_compund} operations the client generally reuses the \textsc{slotid} immediately after receiving the response from the server. But in case of asynchronous read the client should not reuse the \textsc{slotid} after receiving the initial \textsc{nfs4\_ok}, because the client has not received the data yet and thereby the request is not completed. In NFS terms, the requester still has an outstanding request on that \textsc{slotid}, hence it cannot be reused and Figure~\ref{fig:NFSSlotreuse} depicts this scenario. We need to modify the implementation reusing of \textsc{slotid} in case of asynchronous read. This idea can be depicted by the Figure~\ref{fig:NFSSlotreuse}. If the slot is reused before receiving the callback then this causes server to delete the previous entry from the reply cache. In the mean time if the client sends the same request in case of a failure then the server will process the request again which will violate exactly one semantics model.
	
\subsection{Other Asynchronous operations}

	In this paper we have mentioned designed and implemented asynchronous read operation. Similarly we can implement asynchronous write operation in NFS. As write operation is not completely synchronous as read operation, hence we feel we may not observe similar benefits to asynchronous read. But nevertheless we may observe some interesting results in case of simultaneous writes. 
	
