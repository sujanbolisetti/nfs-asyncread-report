\section{Discussion}

	The NFS server on receiving the asynchronous read operation, the worker thread has to reply to the client with \textsc{nfs4\_ok} indicating that it has received the request. The server will then perform the actual read and reply to the client asynchronously using callbacks. The worker thread which receives the request of asynchronous read creates a new thread for performing this actual read operation and then sending reply to the client using callbacks. This implementation is not scalable because as the number of requests increases to some thousands simultaneously we will end up creating thousand odd threads. On a system with average RAM this can lead to a memory crash. We have then decided to use NFS-Ganesha's request flow for performing the actual read operation and sending the data using callback to the client. NFS Ganesha creates a fixed number of pool of threads during the start of the server and this can be configurable by the user. The threads are classified as worker threads and dispatcher threads. Worker threads are responsible for processing the request  and as these threads are in fixed number, this design does not suffer from memory crash problem. This design is also scalable as we can increase the number of threads to run in order to handle more number of requests simultaneously. 