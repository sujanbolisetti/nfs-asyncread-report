\section{Discussion}
\subsection{Exactly once semantics with Asynchronous Read}

	NFSv4.1 supports exactly once semantics model. By exactly once we mean that NFSv4.1 will be able to guarantee that every operation is executed exactly once. This helps the server in identifying the duplicate requests. NFS server will perform this identification with the help of \textsc{slotid} and \textsc{seqid} that will be part of each request. If the server sees another request with the same \textsc{slotid} and \textsc{seqid} that it has seen before, then it will search in  its reply cache and reply to the client. These requests are categorized as duplicate requests and server will not process them again. This obeys exactly once semantics model.In normal \textsc{nfs4\_compound} operations the client generally reuses the \textsc{slotid} after receiving the response from the server on the request associated with that \textsc{slotid}. But in case of asynchronous read the client should not reuse the \textsc{slotid} after receiving the initial \textsc{nfs4\_ok}, because the client has not received the data yet and thereby the request is not completed. In NFS terms, the requester still has an outstanding request on that \textsc{slotid}, hence it cannot be reused and Figure~\ref{fig:NFSSlotreuse} depicts this scenario and our proposed idea to make asynchronous read obey exactly once semantics.