\section{Evaluation}

	We have ran our experiments by running the NFS Ganesha server on Amazon EC2 and Pynfs client on the local machine. We have decided to run our experiments on EC2 as we wanted to evaluate our asynchronous read operation in an environment closer to the real world scenarios. As usually in the real world clients and servers will be on different networks and in this way we are even considering the network latency involved in sending and receiving the requests. We have evaluated asynchronous read operation based on two metrics, one is the completion time and the other is the throughput. We have compared our results with the normal read request.The data size we read in performing normal read and asynchronous read are 512 bytes, 1KB, 1MB and 2MB. The below graph in Fig2 shows the results of these operations.
	

	From the above graph we can observe that there is no/minute overhead involved in performing asynchronous read operation compared to normal read operation. So this result suggests that the  we can perform asynchronous read instead of normal read in all the scenarios where client does not  require data to continue its operations as there is no overhead. 

	The next metric we have evaluated is throughput achieved when using asynchronous read operation. We have observed that the clients can achieve better throughput on using asynchronous compared to normal read. The below graph shows the results of our evaluation.


	As we can observe from the graph the throughput gain compared to normal read will be proportional to the size of the data we are reading. When these asynchronous read requests are batched we can even observe much higher throughput in case of asynchronous read.  



