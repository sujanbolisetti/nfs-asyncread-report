\section{Implementation of Asynchronous Read in Pynfs}

Currently we have implemented asynchronous read using Pynfs in NFS version 4.1. We generated the request and response structures for our asynchronous read using the External Data Representation (XDR) \cite{XDR}. External Data Representation (XDR) is a standard data serialization format. XDR allows the data to be transferred seamlessly between different kinds of systems. Let us now look at each of these structures in detail.

\begin{lstlisting}
struct ASYNC_READ4args{
	uint64_t  reqId; 
	stateid4  stateid; 
	offset4	  offset; 
	count4	  count;
	uint32_t  timeout; 
};
\end{lstlisting}

\noindent\textit{ASYNC\_READ4args} : The client sends these argument structure as part of the initial request to the server. Detailed explanation of the arguments in \textit{ASYNC\_READ4args} will follow. 
\hfill \break \newline
\noindent\textit{reqId} : Reqid is a 64 bit integer, generated by the client. This is generated based on the requested \textit{filehandle}, processId/threadId and current\_timestamp and will uniquely identify every request. The server passes the same request id to the client as part of the callback along with the requested file data. The client then uses this  \textit{reqId} to identify the owner of the request. This owner might be a thread or a process. Note that \textit{reqId} is opaque to the server.
\hfill \break \newline
\noindent\textit{stateid} : Stateid is a 128-bit quantity returned by a server in the initial open request. It uniquely defines the open and locking state provided by the server for a specific open or lock owner for a specific file. We are using \textit{stateid} on the server side to check the client share/delegation access on the requested file. 
\hfill \break \newline
\noindent\textit{offset} : Offset (from the start of the file) at which the read has to start in the file. 
\hfill \break \newline
\noindent\textit{count}: Number of bytes to read from the requested file.
\hfill \break \newline
\noindent On receiving the request from the client, the server performs the initial checks. Then the server responds with \textsc{ASYNC\_READ4res} to the client. Now we will explain the status passed as part of \textsc{ASYNC\_READ4res} to the client.  

\begin{lstlisting}
struct ASYNC_READ4res{
	nfsstat4	  status;
};
\end{lstlisting}
\textit{status} : Indicates the status corresponding to initial permission checks on the requested file.
On receiving the asynchronous read request on the server side, we are checking the client's share/delegation on the requested file. On success we will return \textsc{nfs4\_ok}.  If the checks fail, we are returning appropriate error status.
\hfill \break \newline
\noindent Let us now understand the structure used by the server when making a callback to the client. The server passes \textit{CB\_ASYNC\_READ4args} as part of the request to the client. We will now look at each of the arguments in detail. 
\begin{lstlisting}
union CB_ASYNC_READ4args 
	switch(nfsstat4 status){
	case NFS4_OK:
	 CB_ASYNC_READ4argsok argok4;
 	default:
		void;
};
\end{lstlisting}

\noindent The first argument in the callback request to the client is the status. If the data is fetched successfully from the local file system, the status will be \textsc{nfs4\_ok} . In case of an error, an appropriate error status will be sent to the client. If the data is fetched successfully, a second argument of type \textit{CB\_ASYNC\_READ4argsok} will also be passed in the callback to the client. Otherwise, only status will be passed. A detailed explanation of each of the fields in \textit{CB\_ASYNC\_READ4argsok} will follow.

\begin{lstlisting}
struct CB_ASYNC_READ4argsok{
	uint64_t	 reqId;
	bool		 eof;
	opaque	data<>;
};
\end{lstlisting}

\noindent\textit{reqId}: The \textit{reqId} received by the server from the client during the initial asynchronous read request. This is used by the client to identify the owner of the request.
\hfill \break \newline
\noindent\textit{eof} : A boolean value indicating if the end of the file has been reached.
\hfill \break \newline
\noindent\textit{data} : Requested file data.
\hfill \break \newline
\noindent On receiving the callback request from the server, client forwards the data to the respective owner based on the \textit{reqId}. Then the client responds with \textit{CB\_ASYNC\_READ4res} to the server.
 
\begin{lstlisting}
struct CB_ASYNC_READ4res{
	nfsstat4	 status;
};
\end{lstlisting}

\noindent\textit{status} : Indicates the asynchronous read callback status from the client. On success, status will be \textsc{nfs4\_ok} else the corresponding error message will be passed to the server.




